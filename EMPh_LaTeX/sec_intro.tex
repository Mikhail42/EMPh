\section*{Введение}
% http://tex.stackexchange.com/questions/74439/table-of-contents-incorrect-page-numbering
\phantomsection
\addcontentsline{toc}{section}{Введение}
\label{sec:intro}

В данной работе решается задача теплопроводности на сфере. Изменение температуры происходит за счет поглощения энергии верхней полусферой --- аналогичная ситуация возникает, например, при нагреве Солнцем поверхности Земли.

В первом разделе исходная задача сводится к разложению искомой функции в ряд по ортогональным функциям --- полиномам Лежандра. Это позволяет рассматривать счетное множество независимых уравнений. В конце раздела приводится явное разложение функции температуры, что позволяет найти решение в численном виде. 

Во втором разделе проверяется сходимость ряда и оценивается его остаток в зависимости от количества слагаемых ряда. В результате, можно по заранее заданной точности определить число слагаемых, а значит --- решить задачу с заданной точностью с использованием ЭВМ. 

В третьем разделе приводится код программы, необходимый и достаточный для вычисления функции температуры. В нем же указывается ссылка на полный код проекта (а также на \textit{jar}-архив), скомпилировав (запустив) который можно просматривать график решения. В конце раздела приводятся примеры графиков при заданных параметрах.  
   