\section*{Задание}
%\addcontentsline{toc}{section}{Задание}
\label{sec:solve}
   	Решить задачу теплопроводности на сфере радиуса~$R$. Нагревание происходит за счет поглощения энергии излучения полусферой (см. рис.~\ref{fig:sphere}), причем количество поглощённой энергии определяется косинусом угла между нормалью и направлением распространения излучения. 

   	При моделировании использовать следующую математическую модель:  	
   	\begin{equation}
   		\label{eq: init_equation}
   		\begin{cases}
    	c \pderiv{u}{t} = K \triangle u + \beta f (\overrightarrow{r}),    
    	      \quad  \{x \in   \mathbb{R}^3 \mid \rho(x,0) = R\}, t\geqslant 0; \\
    	u(\theta,\rho,t) \big|_{t=0} = 0.
    	\end{cases}
	\end{equation}  
\begin{explanationx}
	\item[где] $u(\theta, \varphi,t)$ --- функция, описывающая температуру в некоторой точке $(R,\theta,\varphi)$ в момент времени $t$; 
	\item $f$ --- функция, описывающая энергию, поглощенную единицей площади;
	\item $c$ --- положительная константа; 
	\item $K$ --- положительная константа.
\end{explanationx} 

	\begin{figure}[hb!]
 		\begin{center}
 			\includegraphics{/home/misha/latex/EMF/the-3dplot-package.pdf}
			\caption{К условиям задачи}
	 		\label{fig:sphere}
 		\end{center}
	\end{figure} 	
	
	В данной курсовой работе необходимо:
	\begin{enumerate}
	\item Используя метод разделения переменных, получить решение задачи математической физики в виде ряда Фурье.
	\item Исследовать сходимость ряда. Получить оценку остатка ряда. 
	\item Разработать программу расчёта решения задачи с требуемой точностью. (Если необходимо, то использовать метод численного интегрирования для расчёта коэффициентов ряда. При этом следует контролировать погрешность численного интегрирования.)   
	\item Исследовать качество полученной аналитической оценки остатка ряда, используя вычислительный эксперимент. 
	\end{enumerate}